\documentclass[12pt, a4paper]{article}

\usepackage{comment}
\usepackage{ragged2e}
\usepackage{amsmath}
\usepackage{dcolumn}
\usepackage{booktabs}
\usepackage{pdflscape}
\usepackage{graphicx}
\usepackage{placeins}
\usepackage{dcolumn}
\usepackage{xcolor}
\usepackage{booktabs}
\linespread{1.5}
\usepackage{subcaption}
\usepackage{amsmath}
\usepackage{hyperref}
\usepackage{multirow}
\usepackage{tikz}
\usepackage[title]{appendix}
\usetikzlibrary{decorations.pathreplacing}
\usepackage{booktabs}
\usepackage{tabularx}
\usepackage{datatool}
\hypersetup{
	colorlinks=false,
	linkcolor=black,
	filecolor=black,      
	urlcolor=black,
	citecolor = blue
}

\usepackage{natbib}
\usepackage{xepersian}
\settextfont{XB Zar}
\setdigitfont{XB Zar}

\def\sym#1{\ifmmode^{#1}\else\(^{#1}\)\fi}

\title{مالکان نهایی و هم‌زمانی بازده شرکت با بازار}
%\subtitle{}
\author{
	مرتضی آقاجان‌زاده
	 \sym{*} 
	\qquad 
	مهدی حیدری 
	\sym{*} 
	 \\
	\sym{*} 
	\footnotesize  موسسه مطالعات پیشرفته تهران (تیاس) - دانشگاه خاتم
}

\date{
تیر 1400}




\begin{document}

\maketitle

\section{مقدمه}
\section{پیشینه پژوهش}
\section{روش‌شنای پژوهش}
\subsection{هم‌زمانی بازده شرکت}
معولا در ادبیات هم‌زمانی بازده شرکت را با ضریب تعیین برآورد خطی بازده شرکت بر روی بازده بازار و صنعت شرکت محاسبه می‌کنند. هر آنچه یک شرکت دارای ضریب تعیین بالاتری باشد، بازده شرکت با بازده بازار و یا صعنت هم زمانی بالاتری دارد. 
با توجه به مقاله 
پیتروسکی و رولستون (2004)
%\lr{\cite{piotroski2004influence} }
به منظور بدست آوردن هم زمانی قیمت سهام، معادله زیر را برای هر شرکت به صورت سالانه برازش می‌کنیم:
	\begin{equation}\label{e1}
		RET_{i,w} = \alpha + \beta_1 MKRET_{w}+ \beta_2 MKRET_{w-1}  + \beta_3 INDRET_{i,w} + \beta_4 INDRET_{i,w-1} 
	\end{equation}
	که در این معادله 
	\lr{$ RET_{i,w} $}
	بازده هفتگی شرکت i در هفته w، 
	\lr{$ MKRET_{w} $}
	بازده بازار برای هفته w و
	\lr{$ INDRET_{i,w} $}
	بازه صنعت شرکت در هفته w می‌باشد که بازده شرکت مورد بررسی از آن کم شده‌است. 
	
	ضریب تعیین بدست آمده از برازش فوق عددیدر بازه یک و منفی یک می‌باشد به همین علت نمی‌توانیم از آن به صورت مستقیم در برآورد‌های آینده استفاده کنیم. به همین منظور از تبدیل لاجستیک استفاده می‌کنیم. در نتیجه متغیر اصلی هم‌زمانی بازده شرکت برابر است با 	
	\begin{equation}
		SYNCH_{i,t} = log(\frac{R^2_{i,t}}{1-R^2_{i,t}})
	\end{equation}
که در این عبارت $  R^2_{i,t}$ ضریب تعیین بدست آمده از برازش معادله 
\ref{e1}
برای شرکت
 \lr{i}
  و در سال
 \lr{ t}
  می‌باشد. در شکل
  \ref{fig:synchtimeseries} 
  سری زمانی هم‌زمانی بازده شرکت‌ها رسم شده‌است.
  
  \begin{figure}[htbp]
  	\centering
  	\includegraphics[width=0.85\linewidth]{SYNCHtimeSeries}
  	\caption{سری زمانی متوسط هم‌زمانی بازده شرکت‌ها }
  	\label{fig:synchtimeseries}
  \end{figure}
  
\section{یافته‌های پژوهش}
\section{نتیجه‌گیری و پیشنهاد‌ها}


	\begin{itemize}
		\item $ \text{Excess} = (\text{cr} - \text{cfr})/\text{cr} $
		\item $ \text{ExcessDiff} = \text{cr} - \text{cfr} $
		\item $ \text{ExcessDummy} = \left\{\begin{array}{ll}
			1 & \text{cr} - \text{cfr}>0\\
			0 & \text{cr} - \text{cfr}\leq 0
		\end{array}\right.  $
	\item $ \text{ExcessHigh} = \left\{\begin{array}{ll}
		1 & \text{Excess}>\text{Median}(\text{Excess})\\
		0 & \text{Excess}\leq \text{Median}(\text{Excess})
	\end{array}\right.  $
	\end{itemize}

\begin{figure}
	\centering
	\includegraphics[width=0.9\linewidth]{t4}
	\label{fig:t4}
\end{figure}

			\begin{table}[htbp]
	\centering
	\resizebox{0.75\textheight}{!}{
		{
\def\sym#1{\ifmmode^{#1}\else\(^{#1}\)\fi}
\begin{tabular}{l*{8}{c}}
\hline\hline
                    &\multicolumn{8}{c}{Synchronicity}                                                                                                                                              \\\cmidrule(lr){2-9}
                    &\multicolumn{1}{c}{(1)}         &\multicolumn{1}{c}{(2)}         &\multicolumn{1}{c}{(3)}         &\multicolumn{1}{c}{(4)}         &\multicolumn{1}{c}{(5)}         &\multicolumn{1}{c}{(6)}         &\multicolumn{1}{c}{(7)}         &\multicolumn{1}{c}{(8)}         \\
\hline
Excess              &                     &      -0.899\sym{**} &      -0.557\sym{*}  &                     &                     &                     &                     &                     \\
                    &                     &     [-3.22]         &     [-2.10]         &                     &                     &                     &                     &                     \\
[1em]
ExcessDiff          &                     &                     &                     &      -0.512         &                     &                     &                     &                     \\
                    &                     &                     &                     &     [-1.61]         &                     &                     &                     &                     \\
[1em]
ExcessDummy         &                     &                     &                     &                     &     -0.0900         &                     &                     &                     \\
                    &                     &                     &                     &                     &     [-0.66]         &                     &                     &                     \\
[1em]
ExcessHigh          &                     &                     &                     &                     &                     &      -0.175         &                     &                     \\
                    &                     &                     &                     &                     &                     &     [-1.34]         &                     &                     \\
[1em]
position            &                     &                     &                     &                     &                     &                     &     -0.0959\sym{*}  &                     \\
                    &                     &                     &                     &                     &                     &                     &     [-2.53]         &                     \\
[1em]
 $ \ln(\frac{\text{centrality}}{1-\text{centrality}}) $&                     &                     &                     &                     &                     &                     &                     &       0.124\sym{*}  \\
                    &                     &                     &                     &                     &                     &                     &                     &      [2.51]         \\
[1em]
cfr                 &                     &      -0.421         &      -0.173         &      0.0838         &       0.244         &       0.102         &       0.119         &      -0.205         \\
                    &                     &     [-1.17]         &     [-0.53]         &      [0.31]         &      [0.96]         &      [0.38]         &      [0.50]         &     [-0.44]         \\
[1em]
volatility          &    -0.00453         &                     &     -0.0184         &     -0.0168         &     -0.0137         &     -0.0171         &     -0.0182         &       0.809\sym{***}\\
                    &     [-0.27]         &                     &     [-0.95]         &     [-0.87]         &     [-0.69]         &     [-0.86]         &     [-0.92]         &      [3.57]         \\
[1em]
Liquidity           &      -0.206\sym{***}&                     &      -0.191\sym{***}&      -0.192\sym{***}&      -0.196\sym{***}&      -0.195\sym{***}&      -0.195\sym{***}&      -0.235\sym{***}\\
                    &     [-9.33]         &                     &     [-6.17]         &     [-6.29]         &     [-6.29]         &     [-6.31]         &     [-6.30]         &     [-3.59]         \\
[1em]
Size                &     -0.0873\sym{**} &                     &     -0.0952\sym{*}  &     -0.0917\sym{*}  &     -0.0853\sym{*}  &     -0.0879\sym{*}  &      -0.101\sym{*}  &      -0.197         \\
                    &     [-3.03]         &                     &     [-2.17]         &     [-2.09]         &     [-2.01]         &     [-2.06]         &     [-2.25]         &     [-1.80]         \\
[1em]
leverage            &      -0.104         &                     &      -0.281\sym{*}  &      -0.291\sym{*}  &      -0.286\sym{*}  &      -0.273\sym{*}  &      -0.334\sym{**} &      -0.420         \\
                    &     [-1.79]         &                     &     [-2.38]         &     [-2.50]         &     [-2.47]         &     [-2.35]         &     [-2.77]         &     [-1.68]         \\
[1em]
 $ \ln(NIND) $      &      -0.138         &                     &      -0.522         &      -0.526         &      -0.567         &      -0.585         &      -0.602         &      -0.931         \\
                    &     [-0.36]         &                     &     [-0.55]         &     [-0.55]         &     [-0.59]         &     [-0.61]         &     [-0.64]         &     [-0.46]         \\
\hline
Industry Dummy      &         Yes         &         Yes         &         Yes         &         Yes         &         Yes         &         Yes         &         Yes         &         Yes         \\
Year Dummy          &         Yes         &         Yes         &         Yes         &         Yes         &         Yes         &         Yes         &         Yes         &         Yes         \\
Observations        &        2550         &        1116         &         978         &         978         &         978         &         978         &         978         &         333         \\
$ R^2 $             &       0.357         &       0.444         &       0.479         &       0.478         &       0.477         &       0.477         &       0.479         &       0.420         \\
\hline\hline
\multicolumn{9}{l}{\footnotesize \textit{t} statistics in brackets}\\
\multicolumn{9}{l}{\footnotesize \sym{*} \(p<0.05\), \sym{**} \(p<0.01\), \sym{***} \(p<0.001\)}\\
\end{tabular}
}

		\label{tab:synchronicityt4}	
	}
\end{table}


	\begin{figure}
		\centering
		\includegraphics[width=0.9\linewidth]{t5}
		\label{fig:t5}
	\end{figure}

	\begin{table}[htbp]
		\centering
		\resizebox{0.75\textheight}{!}{
			{
\def\sym#1{\ifmmode^{#1}\else\(^{#1}\)\fi}
\begin{tabular}{l*{6}{c}}
\hline\hline
            &\multicolumn{6}{c}{Synchronicity}                                                                                                  \\\cmidrule(lr){2-7}
            &\multicolumn{1}{c}{(1)}         &\multicolumn{1}{c}{(2)}         &\multicolumn{1}{c}{(3)}         &\multicolumn{1}{c}{(4)}         &\multicolumn{1}{c}{(5)}         &\multicolumn{1}{c}{(6)}         \\
\hline
Excess      &                     &      -0.609\sym{**} &      -0.450\sym{*}  &                     &                     &                     \\
            &                     &     [-4.51]         &     [-2.62]         &                     &                     &                     \\
[1em]
ExcessDiff  &                     &                     &                     &      -0.435         &                     &                     \\
            &                     &                     &                     &     [-1.86]         &                     &                     \\
[1em]
ExcessDummy &                     &                     &                     &                     &     -0.0318         &                     \\
            &                     &                     &                     &                     &     [-0.60]         &                     \\
[1em]
ExcessHigh  &                     &                     &                     &                     &                     &      -0.203\sym{*}  \\
            &                     &                     &                     &                     &                     &     [-2.81]         \\
[1em]
cfr         &                     &      -0.432         &      -0.223         &     -0.0273         &       0.151         &      -0.100         \\
            &                     &     [-1.99]         &     [-0.76]         &     [-0.10]         &      [0.60]         &     [-0.37]         \\
[1em]
volatility  &       0.575\sym{**} &                     &       1.665\sym{*}  &       1.632\sym{*}  &       1.630\sym{*}  &       1.689\sym{*}  \\
            &      [5.96]         &                     &      [2.77]         &      [2.80]         &      [2.82]         &      [2.78]         \\
[1em]
liquidity   &      -0.208\sym{***}&                     &      -0.227\sym{***}&      -0.228\sym{***}&      -0.234\sym{***}&      -0.223\sym{***}\\
            &     [-9.81]         &                     &    [-20.43]         &    [-20.28]         &    [-17.24]         &    [-18.99]         \\
[1em]
size        &     -0.0769\sym{**} &      0.0638\sym{**} &      -0.108\sym{**} &      -0.107\sym{**} &     -0.0985\sym{*}  &     -0.0998\sym{*}  \\
            &     [-4.05]         &      [4.06]         &     [-4.37]         &     [-4.41]         &     [-3.66]         &     [-3.93]         \\
\hline
Industry Dummy&         Yes         &         Yes         &         Yes         &         Yes         &         Yes         &         Yes         \\
Year Dummy  &          No         &          No         &          No         &          No         &          No         &          No         \\
Observations&        2758         &        1046         &        1046         &        1046         &        1046         &        1046         \\
$ R^2 $     &       0.442         &       0.521         &       0.551         &       0.552         &       0.548         &       0.550         \\
\hline\hline
\multicolumn{7}{l}{\footnotesize \textit{t} statistics in brackets}\\
\multicolumn{7}{l}{\footnotesize \sym{*} \(p<0.05\), \sym{**} \(p<0.01\), \sym{***} \(p<0.001\)}\\
\end{tabular}
}

			\label{tab:synchronicityt5}	
		}
	\end{table}

	{		
	\bibliographystyle{apalike}
	\bibliography{Ref}
}
\end{document}